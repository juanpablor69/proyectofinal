\documentclass{article}
\usepackage[utf8]{inputenc}
\usepackage[spanish]{babel}
\usepackage{listings}
\usepackage{graphicx}
\graphicspath{ {images/} }
\usepackage{cite}

\begin{document}

\begin{titlepage}
    \begin{center}
        \vspace*{1cm}
            
        \Huge
        \textbf{Ideación}
            
        \vspace{0.5cm}
        \LARGE
       Proyecto final - Informática II
            
        \vspace{1.5cm}
            
        \textbf{Juan Pablo Rendón Jiménez}
            
        \vfill
            
        \vspace{0.8cm}
            
        \Large
        Departamento de Ingeniería Electrónica y Telecomunicaciones\\
        Universidad de Antioquia\\
        Medellín\\
        Marzo de 2021
            
    \end{center}
\end{titlepage}

\tableofcontents
\newpage
\section{Objetivo}\label{intro}
Crear la idea para desarrollar un juego para el proyecto final de Informética II, tratando de que en este juego se reuna todo lo aprendido durante el semestre.

\section{Ideacíon} \label{contenido}
Las ideas que se presentan son las siguientes:
\subsection{Urban Shooting}
Es un juego en el que se debe escapar de la policia utilizando armas, vehiculos y en el que se pueden obetener diversos beneficios al pasar misiones dentro del juego.


\subsection{Mision Monkey}
%
Es un juego en el que un mono tiene que sobrepasar varios obstaculos por diferentes lugares en un determinado lapso de tiempo.  Se obtienen recompenzas y tienen varios mapas para 'rescatar' el juego.

\subsection{Snake}
Es un juego en el que una culebrita tiene que recoger distintas recompenzas distribuidas por todo el mapa y que el usuario debe evitar que la culebrita choque contra alguna pared del juego.

\subsection{Sacred Gems}
Es un juego en el que se debe recoger 8 gemas para poder rescatar a un personaje que se encuentra secuestrado en la selva. Cuando se pase cada dificil nivel, se obtiene una gema, asi hasta llevarle las 8 gemas a los secuestradores y rescatar al personaje.

\end{document}
